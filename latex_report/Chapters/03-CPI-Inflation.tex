\chapter{CPI \& Inflation}
\label{cp:cpi-inflation}

\section{Inflation Dynamics}

The removal of the fuel subsidy in 2023 and the unification of the Naira exchange rate regime served as simultaneous macroeconomic shocks. As a result, the ``All Items'' Consumer Price Index (CPI) increased cumulatively by \textbf{381.7\%} between 2015 and 2024.

This inflationary surge reflects both cost-push pressures---primarily from energy and transportation---and imported inflation driven by exchange rate depreciation in a highly import-dependent economy.


\section{Highest Cost Drivers (2024)}

The most severe price increases over the decade were concentrated in essential consumption categories, intensifying food insecurity and eroding household purchasing power:

\begin{itemize}
	\item \textbf{Imported Food (+41.3\%):} Nigeria’s dependence on imported wheat, sugar, and specialty fats has amplified the transmission of currency depreciation into domestic food prices, significantly increasing food insecurity risks.
	
	\item \textbf{General Food (+39.8\%):} Rising domestic logistics costs---particularly diesel---combined with the displacement of farmers in the North-Central ``food belt'' due to conflict have constrained supply (\cite{FEWSNET2024}).
	
	\item \textbf{Transport (+31.7\%):} Elevated fuel prices have increased the ``last-mile'' cost of virtually every good and service sold across the country.
\end{itemize}

\begin{figure}[H]
	\centering
	\includegraphics[width=0.9\textwidth]{figures/cpi_inflation_2024.png}
	\caption{CPI components experiencing the highest cost-of-living increases over the last ten years.}
	\label{fig:cpi-inflation-2024}
\end{figure}

\section{GDP versus CPI: Sectoral Insights}

\begin{itemize}
	\item \textbf{Resilience:} The \textbf{Information \& Communication Technology (ICT)} sector remains resilient. Digital services are largely inelastic; consumers continue to prioritize data and connectivity even as real incomes decline.
	
	\item \textbf{The Squeeze:} \textbf{Manufacturing and Trade} sectors are under significant stress. While headline GDP figures may show marginal volume growth, rising input, energy, and logistics costs are compressing margins beyond levels that consumers can absorb.
	
	\item \textbf{The Beneficiary:} \textbf{Financial Services} often act as a partial hedge against inflation, as rising price levels translate into higher interest rates and foreign exchange revaluation gains for Tier-1 banks.
\end{itemize}
