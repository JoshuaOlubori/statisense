\chapter{Company--Sector Analysis: A Tale of Two Economies}
\label{cp:company-sector}

While a sector may exhibit ``growth'' in GDP terms---which primarily measures output volume---firms operating within that sector may simultaneously experience financial distress due to value erosion.

Some divergence can be observed between sectoral GDP performance and company-level Profit After Tax (PAT), indicating a disconnect between macroeconomic output and microeconomic profitability.

\section{Sectoral Growth versus Company Profitability}

\begin{table}[H]
	\centering
	\caption{Divergence between sectoral GDP trends and company Profit After Tax (PAT), 2024}
	\label{tab:company-sector-divergence}
	\resizebox{\textwidth}{!}{%
	\begin{tabular}{p{3cm} p{3.5cm} p{3.5cm} p{3cm} p{4.5cm}}
		\hline
		\textbf{Company} & \textbf{Respective Sector} & \textbf{Sector Trend (2024)} & \textbf{Company PAT Growth} & \textbf{Relationship Insight} \\
		\hline
		\textbf{Zenith Bank} & Financial \& Insurance & +29.6\% (Boom) & +57\% &
		\textbf{Aligned:} High interest rates and foreign exchange revaluation gains boosted both sector performance and PAT (\cite{ZenithBank2024}). \\[0.3cm]
		
		\textbf{Seplat Energy} & Manufacturing -- Oil \& Gas & +4.8\% (Recovery) & +16.9\% &
		\textbf{Resilient:} Dollar-denominated revenues provide a natural hedge against exchange rate depreciation. \\[0.3cm]
		
		\textbf{MTN Nigeria} & Information \& Communication Technology (ICT) & +5.4\% (Growth) & --192\% (Loss) &
		\textbf{Diverged:} Service usage increased, but USD-denominated debt severely eroded profitability (\cite{MTNNigeria2024}). \\[0.3cm]
		
		\textbf{Nestlé Nigeria} & Manufacturing & +8.6\% (Growth) & --\(\text{₦}164.6\)bn (Loss) &
		\textbf{Distressed:} High input costs could not be passed on to consumers amid weakening purchasing power (\cite{Nestle2024}). \\[0.3cm]

		\textbf{Okomu Oil} & Agriculture & +2.1\% (Steady) & Growth &
		\textbf{Resilient:} Export-linked commodity pricing insulated margins from domestic inflation. \\[0.3cm]
		
		\textbf{Dangote Cement} & Manufacturing & +8.6\% (Growth) & Downward Pressure &
		\textbf{Mixed:} Strong domestic demand supported volumes, but energy costs and FX exposure on spare parts compressed margins. \\[0.3cm]
		
		\textbf{Conoil} & Manufacturing -- Oil \& Gas & +4.8\% (Recovery) & Growth &
		\textbf{Resilient:} Benefited from fuel subsidy removal and downstream price deregulation. \\
		\hline
	\end{tabular}
	}
\end{table}

\begin{figure}[H]
	\centering
	\includegraphics[width=0.9\textwidth]{figures/company_sector_trends.png}
	\caption{Divergence between sectoral GDP growth (red) and company Profit After Tax (blue).}
	\label{fig:company-sector-trends}
\end{figure}

\section{Key Insight}

Sectoral growth does not necessarily translate into improved company-level profitability. Consumers continue to use more data services (MTN Nigeria) and purchase fast-moving consumer goods such as seasoning products (Nestlé Nigeria), yet firms struggle to convert this demand into profit.

This disconnect arises because the cost of capital and critical production inputs---largely denominated in U.S. dollars---has grown faster than domestic purchasing power, resulting in accounting losses despite stable or growing output volumes.

