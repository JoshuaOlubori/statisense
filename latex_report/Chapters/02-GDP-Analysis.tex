\chapter{GDP Analysis: The Shift from Oil to Services}
\label{cp:gdp-analysis}

\section{Overview}

For decades, Oil \& Gas was the heartbeat of the nation, but between 2015 and 2024, the sector has struggled with a \textbf{compound annual growth rate (CAGR) of \(-4.4\%\)}. This decline is attributed to chronic underinvestment and crude oil theft, which cost the nation an estimated 400,000 barrels per day in 2023 (\cite{BusinessDay2024}).

In contrast, the Services sector---led by Finance and Information \& Communication Technology (ICT)---has stepped in to fill the vacuum, though it does not yet provide the same level of foreign exchange liquidity as the oil sector once did.

\section{Top Five Sectors by Growth (2024)}

\begin{enumerate}
    \item \textbf{Financial \& Insurance (29.61\%):} Driven by high policy rates and dollar-denominated asset revaluation.
    \item \textbf{Water Supply \& Waste Management (8.40\%):} Reflects increased urbanization and ongoing utility privatization.
    \item \textbf{Transportation \& Storage (6.54\%):} Recovering from the 2020--2022 downturns despite persistently high fuel costs.
    \item \textbf{Information \& Communication (5.42\%):} Sustained by the ``necessity'' nature of digital activity in modern economic life.
    \item \textbf{Manufacturing -- Oil \& Gas (4.85\%):} Supported by a mild recovery in domestic refining activities.
\end{enumerate}

\begin{figure}[H]
    \centering
    \includegraphics[width=0.85\textwidth]{figures/gdp_rankings_growth.png}
    \caption{Top five fastest growing sectors in 2024.}
    \label{fig:gdp-growth-2024}
\end{figure}

\section{Top Contributors to GDP (2024)}

\begin{enumerate}
    \item \textbf{Agriculture (24.64\%):} The nation’s largest employer and economic safety net.
    \item \textbf{Services -- Trade (17.69\%):} Characterized by high transaction volumes but structurally low margins.
    \item \textbf{Information \& Communication (17.68\%):} Often described as the ``new oil'' of the Nigerian economy.
    \item \textbf{Manufacturing (8.64\%):} Operating under severe pressure from elevated input and energy costs.
    \item \textbf{Financial \& Insurance (6.22\%):} High value-added output despite a relatively small employment footprint.
\end{enumerate}

\begin{figure}[H]
    \centering
    \includegraphics[width=0.85\textwidth]{figures/gdp_rankings_contribution.png}
    \caption{Top five GDP contributors by average share (2015--2024).}
    \label{fig:gdp-contribution-avg}
\end{figure}

\section{Best and Least Performers: Ten-Year Overview}

\begin{itemize}
    \item \textbf{Best (Growth):} \textbf{Financial and Insurance}, with the strongest average growth rate over the 2015--2024 period.
    \item \textbf{Best (Contribution):} \textbf{Agriculture}, maintaining the highest average share of total GDP.
    \item \textbf{Least (Growth):} \textbf{Manufacturing -- Oil \& Gas}, which experienced persistent contraction over the decade.
    \item \textbf{Least (Contribution):} \textbf{Administrative \& Support Services}, contributing the smallest average share of GDP.
\end{itemize}

\begin{figure}[H]
    \centering
    \includegraphics[width=0.85\textwidth]{figures/gdp_growth_extremes.png}
    \caption{Comparison of the most and least successful sectors by average growth, 2016--2024.}
    \label{fig:gdp-growth-extremes}
\end{figure}