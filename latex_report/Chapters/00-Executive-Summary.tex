

\pdfbookmark[1]{Executive Summary}{executive-summary}
\chapter*{Executive Summary}

\section{The ``K-Shaped'' Economic Divergence}
According to Investopedia.com, a K-shaped recovery ``occurs when, following a recession, different parts of the economy recover at different rates, times, or magnitudes'' (\cite{Investopedia2024}). The data tells a similar story in the Nigerian economy: while the financial and digital economies are booming, the real sector---manufacturing, telecommunications, and consumer goods---is contracting. This is driven by aggressive currency devaluation, the removal of the petrol subsidy, and persistent security challenges in the agricultural belt, which have displaced over 3.3 million people as of 2024 (\cite{IDMC2024}).

\section{Key Highlights}

\begin{enumerate}
\item \textbf{Financial Sector Boom:} The Financial \& Insurance sector has decoupled from the broader economy, recording a \textbf{29.6\% YoY growth in 2024}. This is largely a result of high-interest-rate environments and foreign exchange (FX) revaluation gains.
\item \textbf{Cost of Living Crisis:} Inflation has become a chronic issue. In 2024, \textbf{Imported Food inflation reached 41.3\%}. The average Nigerian's diet is being affected by exchange rate volatility.
\item \textbf{Corporate Profitability Shift:} While \textbf{Zenith Bank} saw profits jump by \textbf{153\% in 2023} (\cite{ZenithBank2024}), industrial giants like \textbf{MTN Nigeria} and \textbf{Nestlé Nigeria} recorded losses, with MTN’s losses widening by \textbf{192\% in 2024} due to massive FX-related liabilities (\cite{MTNNigeria2024, Nestle2024}).
\end{enumerate}

\section{Core Insight}

Policy intervention must shift from targeting mere GDP growth to \textbf{sector-specific stabilization}, particularly focusing on the FX-exposed manufacturing and telecommunications sectors which are the country's primary non-oil employers.

\MediaOptionLogicBlank
